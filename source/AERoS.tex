% This is samplepaper.tex, a sample chapter demonstrating the
% LLNCS macro package for Springer Computer Science proceedings;
% Version 2.20 of 2017/10/04
%
\documentclass[runningheads]{llncs}
%
\usepackage{multirow}
\usepackage{graphicx}
\usepackage{textcomp}
%\usepackage{setspace}
% Used for displaying a sample figure. If possible, figure files should
% be included in EPS format.
%
% If you use the hyperref package, please uncomment the following line
% to display URLs in blue roman font according to Springer's eBook style:
% \renewcommand\UrlFont{\color{blue}\rmfamily}

% Added by Dhaminda
%\setlength{\belowcaptionskip}{-20pt}

\begin{document}
	%
	\title{AERoS: Assurance of Emergent Behaviour for use in Autonomous Robotic Swarms}%\thanks{....}}
%
\titlerunning{Assurance of Emergent Behaviour for use in Autonomous Robotic Swarms}%{Abbreviated paper title}
% If the paper title is too long for the running head, you can set
% an abbreviated paper title here
%
\author{Dhaminda B. Abeywickrama\inst{\dag}\inst{1}\orcidID{0000-0002-4423-0284} \and
	James Wilson\inst{\dag}\inst{1}\orcidID{0000-0002-0758-6732} \and
	Suet Lee\inst{1}\orcidID{2222--3333-4444-5555} \and
	4th Author\inst{1}\orcidID{2222--3333-4444-5555} \and
	5th Author\inst{1}\orcidID{2222--3333-4444-5555} \and
	6th Author\inst{1}\orcidID{2222--3333-4444-5555} \and
	7th Author\inst{1}\orcidID{2222--3333-4444-5555} \and
	8th Author\inst{1}\orcidID{2222--3333-4444-5555} \and
	9th Author\inst{1}\orcidID{2222--3333-4444-5555}\thanks{\dag D. B. Abeywickrama and J. Wilson contributed equally.}}
%
\authorrunning{D. B. Abeywickrama and J. Wilson et al.}
% First names are abbreviated in the running head.
% If there are more than two authors, 'et al.' is used.
%
\institute{University of Bristol, Bristol, UK\\
	\email{\{firstname.lastname\}@bristol.ac.uk}}
%
\maketitle              % typeset the header of the contribution
%
\begin{abstract} 
The overall behaviours of a swarm are not explicitly engineered in the system, but they are an emergent consequence of the interaction of individual agents with each other and their environment. 
This emergent functionality poses a challenge to ensure their \emph{assurance}, such as safety. 
The main contribution of this paper is a process for the safety assurance of the emergent behaviour for use in autonomous robotic swarms called \emph{AERoS}, following the AMLAS guidance for machine learning-based systems. 
We explore our proposed guidance using illustrative examples taken from a public cloakroom case study.
	
\keywords{Assurance \and Safety \and Emergent behaviour \and Guidance \and Swarms.}
\end{abstract}


\section{Introduction}\label{introduction}
Swarm robotics provides an approach to the coordination of large numbers of robots, which is inspired from the observation of social insects \cite{Sahin2005}. 
The functionality of a swarm is emergent, and evolves based on the capabilities of the robots and the numbers of robots used. 
The overall behaviours of a swarm are not explicitly engineered in the system, but they are an emergent consequence of the interaction of individual agents with each other and the environment \cite{Abeywickrama2022}, and this poses a challenge to ensure their \emph{assurance}. 

According to the ISO standard for systems and software engineering vocabulary~\cite{ISO24765:2017}, \emph{assurance} is defined as ``all the planned and systematic activities implemented within the quality system, and demonstrated as needed, to provide adequate confidence that an entity will fulfil requirements for quality''. 
Assurance tasks comprise conformance to standards, verification and validation (V\&V), and certification, and assurance criteria for autonomous systems (AS) include both functional and non-functional requirements such as safety~\cite{Cheng2014}. 

A key limitation of existing standards and regulations of AS is that they do not accommodate the adaptive nature of AS with evolving functionality~\cite{Fisher2020}. 
They are either implicitly or explicitly based on the V\&V model, which moves from requirements through design onto implementation and testing before deployment~\cite{Jia2021}. 
However, this model is unlikely to be suitable for systems with the ability to adapt their functionality in operation; e.g.\ through interaction with other agents and the environment, as is the case with swarms. 
ISO standards have been developed for the service robotics sector (non-industrial) (e.g.\ ISO 13482, ISO 23482-1, ISO 23482-2), and the industrial robotics sector (e.g.\ ISO 10218-1, ISO 10218-2, ISO/TS 15066) \cite{Abeywickrama2022}. 
However, although these industry standards focus on ensuring assurance of robots at the individual-level, they do not cover safety or any other extra-functional property at the \emph{swarm}-level that may arise through emergent behaviour (EB). %, which provides motivation for our study.

The main contribution of this paper is a process or guidance for the safety assurance of EB for use in autonomous robotic swarms (AERoS), following the AMLAS guidance~\cite{Hawkins2021} proposed for machine learning systems. 
AERoS covers six EB lifecycle stages: safety assurance scoping, safety requirements elicitation, data management, model EB, model verification, and model deployment. 
We illustrate the AERoS process using examples taken from a public cloakroom case study. 
Let us consider a company that organises bespoke events with 50 to 10000 attendees~\cite{Jones2020}. 
The company automates their cloakroom by deploying a swarm of robots to assist attendees (customers) to deposit, store, and deliver their belongings (e.g. jackets) \cite{Jones2020}. 
As the swarm operates in a public setting, the system must prioritize on public safety. 

The rest of the paper is organised as follows. 
In Section~\ref{relatedwork}, we provide key related works to our study. Section~\ref{framework} discusses the six stages of the AERoS process. Finally, Section~\ref{discussion-conclusions} provides a brief discussion and concludes the paper. 

\vspace{-2ex}
\section{Related Work}\label{relatedwork}%\label{background-relatedwork}
\vspace{-2ex}
An AS with evolving functionality is considered to follow a different, much more iterative life-cycle compared to the conventional V\&V model. 
Thus, there is a need for new standards and assurance processes that extend beyond design time and allow continuous certification at runtime~\cite{Rushby2008}. 
In this context, there have been several standards and guidance introduced by several industry committees and research groups. In 2016, the British Standards Institution introduced the \emph{BS 8611} standard that provides a guide to the ethical design and application of robots and robotic systems. % \cite{BS8611}. 
%Then, IEEE initiated the development of a series of standards to address autonomy, ethical issues, transparency, data privacy and trustworthiness (for e.g. IEEE P7001~\cite{IEEE-P7001}, P7007, P7010). 
Then, IEEE through its Global Initiative on Ethics of Autonomous and Intelligent Systems initiated the development of a series of standards to address autonomy, ethical issues, transparency, data privacy and trustworthiness (for e.g. IEEE P7001~\cite{IEEE-P7001}, P7007, P7010). 
There are several standards and guidance related to machine learning in aeronautics, automotive, railway and industrial domains \cite{Kaakai2022}, e.g. Assurance of Machine Learning for use in Autonomous Systems (AMLAS) process~\cite{Hawkins2021}, European Union Aviation Safety Agency (EASA) concept paper, DEpendable and Explainable Learning (DEEL) white paper, Aerospace Vehicle System Institute (AVSI) report, Laboratoire National de Métrologie et d'Essais (LNE) certification, and UL 4600 standard. 

\emph{AMLAS} provides guidance on how to systematically integrate safety assurance into the development of the machine learning components based on offline supervised learning~\cite{Hawkins2021}. 
%AMLAS provides an explicit and structured safety case that the system is safe to operate in its intended context of use. 
AMLAS contains six stages, and the assurance activities are performed in parallel to the development of machine learning component. 
The process is iterative by design and feedback is used to update previous stages. 
However, AMLAS specifically targets safety assurance of a machine learning model for a single system, and not for a collective as considered in our study. 
Also, none of the aforementioned standards and guidance target swarm systems. 

\vspace{-2ex}
\section{AERoS Process}\label{framework}
\vspace{-2ex}
In this section, we discuss the six main stages of the AERoS process targeting swarm systems. For each stage, we describe its inputs and outputs, the main assurance activities, and their associated artefacts.
%In this study, we look at inherent swarm qualities and the adaptation that arises from these qualities. 
%In a swarm, individual robots execute simple behaviours or algorithms, and these simple behaviours when performed by a large number of agents build to form \emph{emergent behaviours} (EB).  
%
%We will consider the adaptation of individual agents using machine learning in future work. 
%At present, our study does not consider machine learning components of individual agents. 
%The adaptivity provided by this emergence requires us to provide \emph{assurance} (e.g. safety assurance). 
%In this study, we investigate inherent swarm qualities and the adaptation that arises from these qualities. 
%%%%These simple behaviours performed by large numbers of agents build to emergent behaviours.
%%%%For this, we will look into developing individual behaviours which when combined create an adaptive emergence. 
%The adaptivity provided by this emergence requires us to provide \emph{assurance} (e.g. safety assurance). 
%At present, our study does not consider machine learning components of individual agents. 
%Instead, these will be considered as part of future work once AMLAS has been applied to the swarm.
%However, we may consider this later once AMLAS has been applied to the swarm.

\vspace{-2ex}
\subsection{Stage 1: EB Safety Assurance Scoping} \label{framework-stage1}
Stage 1 contains two activities which are performed to define the safety assurance scope for the swarm (see Fig.~\ref{amlas-a-stage1}). In a swarm, individual robots execute simple behaviours, and these simple behaviours when performed by a large number of agents build to form EB.  
\vspace{-2ex}
\subsubsection*{Activity 1. Define Assurance Scope for the EB Description and Expected Output}
The goal of Activity 1, which has four inputs [A--D], is to define the safety assurance scope for the EB description and expected output. 
The output of this activity is the Safety Requirements Allocated to the Swarm [E]. The requirements defined in this stage are independent of any EB technique or metric, which reflects the need for the swarm system to perform safely regardless of the deployed technology.

\begin{figure}
	\vspace{-4ex}
	\centering
	\begin{minipage}{.5\textwidth}
		\centering
		\includegraphics[width=0.99\textwidth]{figures/AMLAS-STAGE-1-V4.png}
		\vspace{-2ex}
		\caption{Stage 1: AERoS EB assurance scoping process.}
		\label{amlas-a-stage1}
	\end{minipage}%
	\begin{minipage}{.5\textwidth}
		\centering
		\includegraphics[width=0.99\textwidth]{figures/stage1-systema-v2.png}
		\vspace{-2ex}
		\caption{System description.}
		\label{system-description}
	\end{minipage}%
	\vspace{-4ex}
\end{figure}

\emph{[A] System Safety Requirements:}
The system safety assessment process generates the safety requirements of the swarm, and it covers identification of hazards (e.g. blocking of critical paths in the cloakroom) and risk analysis.
Fig.~\ref{failure-events} illustrates how individual robot failure propagates through the neighbourhood to swarm-level hazards: we can then derive safety requirements in the form of concrete failure conditions at the level of the whole swarm which capture, implicitly, all levels of the swarm. 
Although this has been illustrated as a simplified linear chain of events, in reality this represents a complex sequence which can be difficult to distil into distinct events and cause. 

\emph{[B] Environment Description:}
It is essential to consider the system environment when allocating safety requirements to the swarm. 
In the cloakroom, a swarm of robotic agents collects and delivers jackets, which are stored in small box-like containers. 
The agents are required to navigate a public space between collection and delivery points. They use local communication, perception, and data to form an emergent system of navigation which allows them to easily traverse the public space. 

\emph{[C] System Description:}
In the cloakroom, we can consider three inputs: sensor availability, neighbourhood data, and swarm parameters. 
The \emph{sensors} available to agents can be: cameras, Bluetooth communication devices, and light detection and ranging systems (see Fig.~\ref{system-description}). 
The neighbourhood data of the swarm can be specified through the communication systems available to agents, in this case Bluetooth. 
Through the use of this short range communication, we can assume agents have access to neighbourhood data, such as: approximate position of local agents, current behaviour statuses, an approximate history of box movement, and the amount of time deployed. 
As for the swarm-level parameters, we can consider options specified by a user, i.e. the number of agents deployed, maximum speed of agents, and the number of agents permitted to be active at a time.
Once defined, the three inputs are then fed to the individual agents to instruct their behaviour. This behaviour enacted by multiple agents then produces a swarm-level EB as the individuals interact with one another and their environment.


\begin{figure}
	\vspace{-4ex}
	\centering
	\begin{minipage}{.5\textwidth}
		\centering
		\includegraphics[width=0.99\textwidth]{figures/stage1-failureevents-v3.png}
		\vspace{-2ex}
		\caption{Failure conditions in a swarm adapted from DO-178C and AMLAS.}
		\label{failure-events}
	\end{minipage}
	\begin{minipage}{.45\textwidth}
		\centering
		\includegraphics[width=0.99\textwidth]{figures/stage1-argumentpattern-v3.png}
		\vspace{-2ex}
		\caption{EB Safety assurance scoping argument pattern.}
		\label{stage1-ap}
	\end{minipage}	
	\vspace{-5ex}
\end{figure}

\vspace{-1ex}
\emph{[D] EB Description and Expected Output:}
By ``expected output", we refer to the gains that can arise from the system by deploying multiple agents. 
In the cloakroom use case, the output is a collaborative system capable of collecting, sorting, and redelivering jackets in a public setting. 
To achieve this, the EB of the system needs to be manually engineered with consideration to the available sub-behaviours within an agent and the constraints outlined in the system description.

\vspace{-2ex}
\subsubsection*{Activity 2. Instantiate the EB Assurance Scoping Argument Pattern}
Each stage of the AERoS process includes an activity to instantiate a safety argument pattern based on the evidence and artefacts generated in that stage, following AMLAS~\cite{Hawkins2021}. 
\textit{Argument patterns}, which are modelled using the Goal Structuring Notation, can be used to explain the extent to which the evidence supports the relevant EB safety claims.  
%GSN is widely used graphical notation to capture safety cases.  
In Activity 2, we use the artefacts generated from Stage 1 (i.e. [A–E]) to instantiate the EB Assurance Scoping Argument Pattern ([F] – see Fig.~\ref{stage1-ap}). 
The instantiated argument [G] along with other instantiated arguments resulting from the other five stages of AERoS constitute the safety case for the swarm. 
Due to space limitations, activities to instantiate argument patterns of other stages are not included in this paper.
%Due to space limitations, the descriptions of activities to instantiate argument patterns in other stages of AERoS are not included in this paper.

\vspace{-2ex}
\subsection{Stage 2: EB Safety Requirements Assurance} \label{framework-stage2}
Stage 2 contains three activities (Fig.~\ref{amlas-a-stage2}), which are performed to provide assurance in EB safety requirements for the swarm. 
The scope of this stage is limited to the EB model of the swarm.

%\vspace{-4ex}
\begin{figure}
	\vspace{-4ex}
	\centering
	\begin{minipage}{.55\textwidth}
		\centering
		\includegraphics[trim={0mm -6mm 0mm -6mm},clip,width=\textwidth]{figures/amlas-a-stage2-v3.png}
		\vspace{-2ex}
		\caption{Stage 2: AERoS EB safety requirements assurance.}% process.}
	\label{amlas-a-stage2}
\end{minipage}%
\begin{minipage}{.45\textwidth}
	\centering
	\includegraphics[trim={30mm 25mm 45mm 30mm},clip,width=0.9\textwidth]{figures/3Dsim.png}
	\vspace{-2ex}
	\caption{3D simulation created to validate several EB safety requirements.}
	\label{3Dsim}
\end{minipage}
\vspace{-4ex}
\end{figure}

\vspace{-2ex}
\subsubsection*{Activity 3. Develop EB Safety Requirements}

The required input to Activity 3 in Stage 2 is the Safety Requirements Allocated to the Swarm [E]. 
We define EB safety requirements to control the risk of swarm-level hazards by taking into account the system architecture defined and operating environment. 

In the swarm context, we consider four types of requirements: \emph{performance}, \emph{adaptability}, \emph{human safety}, and \emph{environment}. 
In particular, the environment requirements capture the need for the system to be robust to variation in the operative space.
We consider several performance safety metrics under each requirements category: 
(i) performance: low impact and high impact collisions; 
(ii) adaptability: percentage of swarm stationary outside of the delivery site, number of stationary agents, time since last agent moved; 
(iii) human-safety: velocity or average velocity of agents, swarm size, rate of human encountered, proximity to humans;
(iv) environment: sum of objects/$m^2$.
As the swarm system is composed of many agents, there is potential for a large number of faults to occur at any given time. This motivates three further sub-categories for each of performance, adaptability and human-safety requirements: \emph{faultless operations}, \emph{failure modes (graceful degradation)}, and \emph{worst case}. 
\emph{Graceful degradation} refers to the acceptable level of faults, their impact, and how the system should react when those faults are introduced. \emph{Worst case} accounts for the least acceptable impact the system should experience and the means to avoid it. 
A key output of Activity 3 is the EB Safety Requirements [H], which states the safety requirements relating to: performance, adaptability and environment (see Table \ref{tab:reqs}), as well as human safety requirements (see Table \ref{tab:human-s}).

\vspace{-2ex}
\subsubsection*{Activity 4: Validate EB Safety Requirements}

The required input to Activity 4 is the EB Safety Requirements [H].  
The EB safety requirements are validated by both review and simulation.
Firstly, the requirements derived for the cloakroom have been reviewed by a safety-critical systems engineering expert to ensure that the specified EB safety requirements for the swarm will deliver its intended safe operation. Secondly, in simulation we validated all safety requirements specified (excepting RQ3.5) for the cloakroom system using the Gazebo 3D simulator. 
The simulation used is an exact replication of the 4m x 4m lab environment used for hardware implementation (see Fig.~\ref{3Dsim}). 
In \textbf{Activity 5}, the artefacts generated in this stage are used to instantiate the EB Safety Requirements Argument Pattern [I].

%\vspace{-4ex}
\tiny
\begin{table}[!t]
\centering
\begin{tabular}{p{5mm} p{116mm} }
	RQ & \textbf{Performance Requirements}\\
	\hline
	1.1 & The swarm \emph{shall} experience \textbf{$<$ 1 high impact (V $>$ 0.5m/s)} collisions across \textbf{a day} of \textbf{faultless} operation. \\ 
	\hline
	1.2 & The swarm \emph{shall} experience \textbf{$<$ 0.1\%} increase in \textbf{high impact} collisions across \textbf{a days} operation with \textbf{10\% injection} of \textbf{full communication fault} to the swarm.\\ 
	\hline
	1.3 & The swarm \emph{shall} experience \textbf{$<$ 0.1\%} increase in \textbf{high impact} collisions across \textbf{a days} operation with \textbf{50\% injection} of \textbf{half-of-wheels motor faults} to the swarm.	\\	
	\hline
	1.4 & The swarm \emph{shall} experience \textbf{$<$ 2 high impact (V $>$ 0.5m/s)} collisions across \textbf{a day} of \textbf{faulty} operation.  \\		[1ex] 		
	\hline
	& \textbf{Adaptability Requirements}\\
	\hline
	2.1 & The swarm \emph{shall} have \textbf{$<$ 10\%} of its agents \textbf{stationary*} outside of the \textbf{delivery site} at a given time.
	*Assumption: Agents are considered stationary once they have not moved for $>$ \textbf{10 seconds}.
	\\ 
	\hline
	2.2 & All agents of the swarm \emph{shall} move at least every \textbf{100 seconds} if outside of the \textbf{delivery site}. \\ 
	\hline
	2.3 & The swarm \emph{shall} experience $<$ \textbf{10\%} increase in \textbf{number of stationary agents} at any given time with \textbf{50\% injection} of \textbf{half-of-wheels motor faults} to the swarm. \\
	\hline
	2.4 & The swarm agents \emph{shall} experience $<$ \textbf{10\%} increase in \textbf{stationary time} with \textbf{50\% injection} of \textbf{half-of-wheels motor faults} to the swarm.\\ 
	\hline
	2.5 & The swarm \emph{shall} experience $<$ \textbf{10\%} increase in \textbf{number of stationary agents} at any given time \textbf{10\% injection} of \textbf{full communication fault} to the swarm.\\
	\hline
	2.6 & The swarm agents \emph{shall} experience $<$ 10\% increase in \textbf{stationary time 10\% injection} of \textbf{full communication fault} to the swarm. \\	
	\hline
	2.7 & The swarm \emph{shall} have \textbf{$<$ 20\%} of its agents \textbf{stationary*} outside of the \textbf{delivery site} at a given time.
	*Assumption: Agents are considered stationary once they have not moved for $>$ \textbf{10 seconds}. \\
	\hline
	& \textbf{Environmental Requirements}\\
	\hline
	3.1 & The swarm \emph{shall} perform as required in environmental density levels \textbf{0-4 p$_o$ of objects} (sum of boxes and agents per m$^2$) in the environment. %* p$_o$ = sum of objects / m$^2$.
	\\ 
	\hline
	3.2 & The swarm \emph{shall} perform as required when \textbf{floor incline} is \textbf{0-20 degrees}.
	\\ 
	\hline
	3.3 & The swarm \emph{shall} perform as required in a \textbf{dry environment}.
	\\ 
	\hline
	3.4 & The swarm \emph{shall} perform as required in \textbf{smooth-floored environments} with step increases no greater than \textbf{0.5cm}.
	\\ 
	\hline
	3.5 & The swarm \emph{shall} only operate in \textbf{environments where humans have devices that identify the human’s whereabouts} to the swarm agents.
	\\		[1ex] 		
	\hline
\end{tabular}
\caption{\label{tab:reqs}Safety requirements for the cloakroom.}
\vspace{-4ex}
\end{table}
\normalsize

%\vspace{-4ex}
\tiny
\begin{table}[!h]
\centering
\begin{tabular}{p{6mm} p{116mm}}
	RQ & \textbf{Human Safety Requirements} \\
	\hline
	4.1 & The agents in the swarm \emph{shall} travel at speeds of less than \textbf{0.5m/s} when within \textbf{2m} distance of a \textbf{trained human}.
	\\ 
	\hline
	4.2 & The agents in the swarm \emph{shall} travel at speeds of less than \textbf{0.25m/s} when within \textbf{3m} distance of a \textbf{member of the public}.
	\\ 
	\hline
	4.3 & The agents in the swarm \emph{shall} only come within \textbf{2m} distance of a \textbf{human $<$ 10} times collectively across \textbf{1000 seconds} of \textbf{faultless} operations.
	\\ 
	\hline
	4.4 & The swarm \emph{shall} only allow \textbf{$<$ 5 agents} to request intervention from a \textbf{trained human} at a given time
	\\ 
	\hline
	4.5 & A \textbf{trained human} \emph{shall} monitor \textbf{5-20 agents} at a given time.
	\\ 
	\hline
	4.6 & The swarm \emph{shall} only allow \textbf{1 agent} to request input from a \textbf{member of the public} at a given time.
	\\ 
	\hline
	4.7 & A \textbf{member of the public} \emph{shall} receive  information from $<$ \textbf{5 agents} of the swarm at a given time.
	\\ 
	\hline
	4.8 & The swarm \emph{shall} experience \textbf{$<$ 10\%} increase in \textbf{human encounters} across \textbf{1000 seconds} of operation with \textbf{10\% injection} of \textbf{full communication fault} to the swarm. \\
	\hline
	4.9 & The swarm \emph{shall} experience \textbf{$<$ 10\%} increase in \textbf{human encounters }across \textbf{1000 seconds} of operation with \textbf{50\% injection} of \textbf{half-of-wheels motor faults} to the swarm.\\
	\hline
	4.10 & The agents in the swarm \emph{shall} only come within \textbf{2m} distance of a \textbf{human $<$ 20} times collectively across \textbf{1000 seconds} of \textbf{faulty} operations.
	\\		[1ex] 		
	\hline
\end{tabular}
\caption{\label{tab:human-s}Human safety requirements for the cloakroom.}
\vspace{-4ex}
\end{table}   
\normalsize

\vspace{-2ex}
\subsection{Stage 3: Data Management} \label{framework-stage3}
When designing EB, data plays a vital role, though one that differs from the machine learning applications AMLAS was initially designed for. In order to address this difference, the following activities and outputs have been adjusted to take into account the swarm behavioural design process and the added complexities that come with multiple agents interacting with one another. Additionally, we specify requirements based on the environments and training parameters which provide the data for training rather than on the datasets themselves.

%\vspace{-2ex}
\begin{figure*}
\centering
\includegraphics[width=0.75\textwidth]{figures/AMLAS-STAGE-3-V4.png}
\vspace{-2ex}
\caption{Stage 3: AERoS data management process.}
\label{amlas-a-stage3}
\vspace{-4ex}
\end{figure*}

\vspace{-2ex}
\subsubsection*{Activity 6. Define Data Requirements}
In our adaptation of Activity 6, we take the EB Safety Requirements [H] outlined in Stage 2 as an input (see Fig.~\ref{amlas-a-stage3}). These safety requirements guide the data requirements in this activity, feeding into the data specification that we outline here. We split the data requirement outputs into two multi-agent focused requirements: [L.0] Data Type Requirements and [L.1] Data Availability Constraints.

\tiny
\begin{table}[!t]%[h]
\centering
\begin{tabular}{p{0.6cm} p{11.6cm}}
	\textbf{RQ} & \textbf{Relevance Requirements Examples} \\
	\hline
	5.1 & All simulations shall include environments with ranges of incline between 0-20\textdegree.\\
	\hline
	5.2 & All simulations shall be conducted in a dry environment.\\
	% \hline
	% 5.3 & All simulations shall include floors with step increases that are $\leq$ 0.5 cm.\\
	\hline
	& \textbf{Completeness Requirements Examples} \\
	\hline
	6.1 & All simulations shall be repeated to include fault injections representative of full communication faults.\\
	\hline
	6.2 & All simulations shall be repeated a sufficient number of times to ensure results are representative of typical use.\\
	\hline
	6.3 & All simulations shall be repeated in multiple environments representative of those expected in real-world use of the system.\\
	% \hline
	% 6.4 & All simulations shall include sufficient range of robot density levels within the scope of the operational domain.\\ 
	\hline
	& \textbf{Accuracy Requirements Examples} \\
	\hline
	7.1 & All boxes shall only be considered `delivered’, if all four of the boxes’ feet are positioned within the delivery zone.\\
	\hline
	7.2 & All boxes shall only be considered `delivered’, once they are no longer in direct contact with a swarm agent.\\
	\hline
	& \textbf{Balance Requirements Examples} \\
	\hline
	8.1 & All simulations shall be repeated so as to obtain representative evaluations for each possible mode of failure (defined under performance, adaptability and human-safety requirements in Stage 2).\\
	\hline
	8.2 & All simulations shall be repeated equally across all test environments.\\
	\hline
\end{tabular}
\caption{Requirement examples for output [L.0].}
\label{tab:L0_req}
\vspace{-4ex}
\end{table}
\normalsize

%\paragraph*{[L.0] Data Type Requirements}
\emph{[L.0] Data Type Requirements:}
This element focuses on the \emph{relevance}, \emph{completeness}, \emph{accuracy}, and \emph{balance} of the information that will be used to construct the swarm behaviour and will be subsequently used to test the EB of the system prior to its deployment. The \emph{relevance} of the data used in the development of the EB specifies the extent to which the test environment must match the intended operating domain into which the model is to be deployed. The \emph{completeness} of the data specifies the conditions under which we test the behaviour algorithm, i.e. the volume of experiments or tests that will be run, the variety of tests executed, and the diversity of environments expected to be used in the testing process. The aim is to cover a representative sample of conditions for testing. \emph{Accuracy} in this context relates to the parameters defining the performance of the swarm systems primary function. For example, what constitutes a delivery in a logistics scenario, or under what conditions would an area be considered explored in a surveying mission. \emph{Balance} refers to the balance of the trials executed in the testing process of the EB algorithm. By considering balance, we expect the number of tests conducted for failure modes or environments to be justified, ensuring that there is not an unrealistic bias in testing towards a particular scenario. See Table \ref{tab:L0_req} for examples of data requirements relating to relevance, completeness, accuracy, and balance.

\emph{[L.1] Data Availability Constraints:}
With the introduction of multiple agents comes the issue of data availability. Distributed communication is a key feature found in emergent systems. As such, it is crucial to define how much information each agent is expected to hold, how easily data may transfer between agents, and across what range agents should be able to transfer information between one another. Feasible constraints include: (i) \emph{storage capacity: }the swarm agents shall have a maximum of 2 GB of information stored on board at any point in time; (ii) \emph{available sensors:} the swarm agents shall only have access to environmental data deemed feasibly collectable by radially positioned infrared sensors; (iii) \emph{communication range} the swarm agents shall only have access to other agent data when within communications range of 5 meters; and (iv) \emph{operator feedback} the swarm agents shall only share information with non-agents (e.g. operator terminal) when within communications range of 5 meters.

\emph{[M] Data Requirements Justification Report:}
This report acts as an assessment of the data requirements, providing analysis and explanation for how the requirements and constraints (outlined in [L.0] and [L.1]) address the EB Safety Requirements specified in [H].

\vspace{-2ex}
\subsubsection*{Activity 7. Define Swarm Evaluation Requirements}

Taking the outputs [L.0] and [L.1] from Activity 6, the evaluation requirements take into account how the EB of the swarm will be assessed, specifying the testing environment and the metrics comprising the test results.  

\emph{[N] Test Environment:} This takes into consideration the requirements specified in Activity 6, and defines the environment in which the EB will be tested. In most cases this will be multiple simulation environments featuring diverse sets of terrain, environmental conditions and obstacle configurations. There may also be instances in which this test environment is specified as a physical environment operating under laboratory conditions, with a hardware system acting as a test bed to observe designed behaviours.

\emph{[O] Swarm Performance Metrics:} This output is used to quantify how well the system is performing. While there may be multiple performance metrics, these metrics should be defined with respect to the primary function of the swarm system. Metrics that might feature in this output could include: the delivery rate in a logistics scenario, rate of area coverage in an exploration task, or the response time in disaster scenarios.

\emph{[P] Verification Metric:} This metric should be derived from the EB Safety Requirements [H] specified in Stage 2. These metrics are intended to be used as the criteria for success within the verification process. Examples of these metrics and their related safety requirements might include: swarm density which is used in verifying environmental safety specifications such as RQ3.1, maximum collision force experienced by agents, which could be used to verify that the swarm meets safety performance requirements such as RQ1.1 and RQ1.2, or current speed of all agents, a metric relating directly to the human safety requirements RQ4.1 and RQ4.2.

\emph{[Q] Sensing and Metric Assumptions Log:} This log serves as a record of the details and decisions made in activities 6 and 7. It should contain details of the choices made when producing the Test Environment [N], Swarm Performance Metrics [M], and the Verification Metric [P].
\vspace{-2ex}
\subsubsection*{Activity 8. Validate Evaluation Requirements}

Taking into account outputs [N], [O] and [P] from Activity 7, this activity aims to validate these components with respect to the requirements specified in Activity 6. Should any discrepancies exist between the data requirements and the evaluation requirements, they should be justified appropriately and recorded in the output Swarm Evaluation Validation Results [S]. 
The artefacts generated in this stage are used to instantiate the EB Data Argument Pattern [R] in \textbf{Activity 9}.

\vspace{-2ex}
\subsection{Stage 4: Model Emergent Behaviour} \label{framework-stage4}
In the design of an EB algorithm, the challenge is in selecting behaviours at the individual level of the agent which give rise to the desired EB at the swarm-level. 
In the original AMLAS process, Stage 4 focuses on the creation, testing, and instantiation of a machine learning model for a single system with no consideration to EB for a collective. 
In our adaptation of AMLAS for the robot swarm,  we step away from the machine learning paradigm to allow consideration for all possible optimisation algorithms which may attain the target EB.

\begin{figure}[!h]
\centering
\includegraphics[width=0.55\textwidth]{figures/AMLAS-STAGE-4-V5.png}%amlas-a-stage4.png
\vspace{-2ex}
\caption{Stage 4: AERoS model learning process.}
\label{amlas-a-stage4}
\vspace{-4ex}
\end{figure}

\vspace{-2ex}
\subsubsection*{Activity 10. Create EB Algorithm}

The EB algorithm is engineered at the level of the individual agent behaviours for the Test Environment output [N] from Stage 3. The resultant EB must meet the Safety Requirements [H] defined in Stage 2 (see Fig.~\ref{amlas-a-stage4}). 
In the cloakroom case study, the target EB for the swarm must ensure that items are stored and retrieved by individuals whilst meeting all requirements specified. For example, performance requirements RQ1.1 and RQ1.2 specify an upper bound on the low/high impact collisions that a swarm shall experience in a given time frame. These requirements may be fulfilled by constraining the maximum velocity of individual robots or by ensuring that a robot has a camera or infra-red sensors enabling it to detect obstacles. The key output from this activity is the candidate EB for testing.

\emph{[U] Model Development Log:} This should log the rationale in the design process of the EB algorithm, in particular how all specified safety [H] and Data Type Requirements [L.0] have been met given the Data Availability Constraints [L.1].

\vspace{-2ex}
\subsubsection*{Activity 11. Test EB Algorithm}

In this activity, the candidate EB will be tested against the Swarm Performance Metrics [O] produced in Stage 3. Testing ensures that the EB performs as desired with respect to the defined metrics and in the case where performance passes accepted thresholds, the EB Algorithm [V] will be produced as the output of the activity. 

\emph{[X] Internal Test Results:} This output provides a degree of transparency in the testing procedure as the results may be further examined to ensure tests have run correctly. 
In \textbf{Activity 12}, the artefacts generated in this stage are used to instantiate the EB Argument Pattern [W].

\vspace{-2ex}
\subsection{Stage 5: Model Verification} \label{framework-stage5}
%\vspace{-2ex}
\subsubsection*{Activity 13. Verify EB}
The inputs to the verification process are the EB Safety Requirements [H], Verification Scenarios (Test Generation) [P], and EB Algorithm [V] (see Fig.~\ref{amlas-a-stage5}). 
%
The verification method and assessment process within that method will be largely determined by the specifics of the safety requirements. Some safety specifications lend themselves towards certain assessment methods due to the scenarios they prescribe.
%
For example, to assess that the swarm system meets the requirements for performance given a motor fault injection RQ1.3, it may be easier to realise this in physical or simulation-based testing approaches rather than constructing a reliable formal model of robot behaviour given the complex physical dynamics of a faulty wheel.

%
However, when considering the adaptability requirements, a formal, probabilistic-based verification technique of the EB Algorithm [V] is more suitable. For example, in RQ2.1 analysis using a probabilistic finite state machine of the swarm behaviour could identify the dwell period within states. Monitors could be used to observe when agents enter a stationary state, e.g. \texttt{agent\_velocity=0 $\land $  t\_counter  $\ge$ 100}, and identify if time within that state exceeds some fixed value, and ascertain a probabilistic value to this metric.

\emph{[P] Verification Scenario (Test Generation):} In most cases there will be multiple, valid verification scenarios (test cases) applicable for each of the safety specifications. A `good test case' must be \emph{effective} at finding bugs or defects, \emph{efficient} in minimising the number of tests required, use resources \emph{economically} and be \emph{robust} to system changes~\cite{Fewster1999}. 

Verification results from individual assessments (Z) form entries in the verification log (AA). The verification log identifies assessments where assurance of the EB algorithm (V) is acceptable with respect to the safety requirements (H) and can be used as a set of evidence for building an assurance case.%, eg. safety case, regulator, type approval. 
The artefacts generated in this stage are used to instantiate the EB Verification Argument Pattern [BB] in \textbf{Activity 14}.

\vspace{-2ex}
\subsection{Stage 6: Model Deployment} \label{framework-stage6}
\vspace{-2ex}
\subsubsection*{Activity 15. Integrate EB}

With the EB verified, the next step is to take the EB Algorithm [V], System Safety Requirements [A], Environment Description [B], and System Description [C] and integrate the EB with the system to be deployed (see Fig.~\ref{amlas-a-stage6}). 

In this activity, we use the inputs to this stage to educate the implementation of the EB and anticipate errors we might expect in the interactions between agents and the overall EB. Despite the rigorous validation and testing conducted in previous stages, there will still be a gap between the test environment and the intended, everyday use, deployed scenario. The output, [DD] Erroneous Behaviour Log, captures these anticipated gaps between testing and reality and the differences in behaviour that may surface. 

\vspace{-4ex}
\begin{figure}[!h]
\centering
\begin{minipage}{.5\textwidth}
	\centering
	\includegraphics[width=0.99\textwidth]{figures/AMLAS-STAGE-5-V4.png}%amlas-a-stage5.png
	\vspace{-2ex}
	\caption{Stage 5: AERoS verification process.}
	\label{amlas-a-stage5}
\end{minipage}%
\begin{minipage}{.5\textwidth}
	\centering
	\includegraphics[width=0.99\textwidth]{figures/AMLAS-STAGE-6-V5.png}%amlas-a-stage5.png
	\vspace{-2ex}
	\caption{Stage 6: AERoS model deployment assurance process.}
	\label{amlas-a-stage6}
\end{minipage}
\vspace{-4ex}
\end{figure}

\vspace{-2ex}
\subsubsection*{Activity 16. Test the Integration}

Once the initial integration is complete, the physical implementation should undergo additional testing in which the system will be observed in multiple operational scenarios, as specified in [EE].

\emph{[EE] Operational Scenarios:} These operational scenarios should reflect the environment descriptions specified in [B], offering a real-world situations to examine the behaviour of the integrated system. The testing of the integrated system in these true-to-operation environments should be conducted in a safe manner. Ensuring that the entire multi-agent system can be shut down in an emergency, or providing shadow operators for groups of agents, taking over should the swarm behave erroneously. In our use case, an example of [EE] may take the form of a small deployment of agents in a real but controlled storage area.

\emph{[FF] Integration Testing Results:} Results from the integration testing will be reported here, detailing how the system performs against the EB Safety Requirements [H] specified in Stage 2. 
The artefacts generated in this stage are used to instantiate the EB Deployment Argument Pattern [GG] in \textbf{Activity 17}.

\vspace{-2ex}
\section{Discussion and Future Work} \label{discussion-conclusions}
Using AMLAS \cite{Hawkins2021} as a foundation, we have produced the six stage development pipeline AERoS. This pipeline acts as guidance for those looking to construct swarm robot systems, particularly those that exhibit EB through environmental and agent-to-agent interaction. The stages of AERoS break down the design of these systems to ensure that fundamental safety requirements are adhered to, even in instances of system degradation and compounded failures that should be expected, and managed, in swarm robot solutions. We achieve this with an approach that allows for iteration and feedback to the previous stages as issues of safety are encountered and investigated. We combine this iteration with repeated specification at each stage, observing the issue of safety through the lens of: data, modelling/behaviour design, verification, and deployment.

While the focus of the AERoS process is to ensure safety assurance of EB in swarms, the trustworthiness of an AS can be dependent on many factors other than safety, which include consideration of ethics, governance, and regulation of AS design and operation. 
To this end, our future work intends to build on \cite{Porter2022}, and develop ethics requirements for swarm robots around the ethical principles of beneficence, non-maleficence, respect for autonomy, and justice. A few examples of the core considerations to ethics requirements, each corresponding to a principle, are: (i) RQ1: The swarm \emph{shall} provide benefits to stakeholders; (ii) RQ2: The swarm \emph{shall} avoid causing unjustified harm to stakeholders; (iii) RQ3: The swarm \emph{shall} avoid undue constraints on stakeholders' personal autonomy; and (iv) RQ4: The swarm \emph{shall} allow fair treatment of stakeholders. 
In addition to ethical requirements, we intend to introduce regulatory requirements into the consideration of AS specification. 
In particular we have observed the work of Macrae \cite{macrae2021learning}, who proposes: Structural, Organisational, Technological, Epistemic, and Cultural (SOTEC) framework of sociotechnical risk in Autonomous and Intelligent systems. 
Viewing regulatory requirements analysis from a sociotechnical perspective allows us to move away from a purely technical conception of requirements, and helps us design AS that better fit the organisation and operators’ work in which safety considerations are meaningful within the wider system and operational context. 
In future work, we will illustrate the relevance of SOTEC for crafting regulatory requirements for swarms as a safety assurance mechanism.
%
%
%...
%
%The focus of the AERoS process is ensuring safety assurance of EB in swarms. 
%However, the trustworthiness of an AS can be dependent on many other factors and not just safety, which include consideration of ethics, and governance and regulation of AS design and operation. 
%To this end, our study intends to build on \cite{Porter2022}, and develop ethics requirements for swarm robots around the ethical principles of beneficence, non-maleficence, respect for autonomy, and justice. A few examples of the main ethics requirements, each corresponding to a principle, are: (i) RQ1: The swarm \emph{shall} provide benefits to stakeholders; (ii) RQ2: The swarm \emph{shall} avoid causing unjustified harm to stakeholders; (iii) RQ3: The swarm \emph{shall} avoid undue constraints on stakeholders' personal autonomy; and (iv) RQ4: The swarm \emph{shall} allow fair treatment of stakeholders. 
%In addition to ethical requirements, we intend to introduce regulatory requirements into the consideration of AS specification.
%In particular we have observed the work of Macrae \cite{macrae2021learning}, who proposes: Structural, Organisational, Technological, Epistemic, and Cultural (SOTEC) framework of sociotechnical risk in Autonomous and Intelligent systems. 
%Viewing regulatory requirements analysis from a sociotechnical perspective allows us to move away from a purely technical conception of requirements, and helps us design AS that better fit the organisation and operators’ work in which safety considerations are meaningful within the wider system and operational context. 
%We will illustrate the relevance of SOTEC for crafting regulatory requirements for swarms as a safety assurance mechanism.
%We will discusses SOTEC and the regulatory implications for swarms in a cloakroom scenario. 
%and discusses its regulatory implications for autonomous swarms in a cloakroom scenario.
%Based on these considerations, we illustrate the relevance of the SOTEC framework for crafting regulatory requirements for autonomous robotic swarms as a safety assurance mechanism.

%\noindent \textbf{[Ethics and Regulatory requirements]}
%%\subsubsection{Ethics Requirements}\label{ethics-r} 
%\newline Building on \cite{Porter2022}, and adapting it for our purposes, we intend to develop ethics requirements for swarm robots around the ethical principles of beneficence, non-maleficence, respect for autonomy, and justice. For this we will focus on the impact of our case study on members of the public, trained humans, manufacturers, and the company deploying the swarm, hence leaving aside questions about broader societal and environmental implications. A few examples of the main ethics requirements, each corresponding to a principle, are: (i) RQ1: The swarm \emph{shall} provide benefits to stakeholders; (ii) RQ2: The swarm \emph{shall} avoid causing unjustified harm to stakeholders; RQ3: The swarm \emph{shall} avoid undue constraints on stakeholders' personal autonomy. RQ4: The swarm \emph{shall} allow fair treatment of stakeholders.  

%
%These broad requirements should be further specified, and this is where developing ethics matrices is helpful. The matrices allow us to identify benefits, risks of harm, and threats to autonomy and fair treatment caused by the development and deployment of the system. The significance and likelihood of the identified benefits, harms, and threats should then be considered to decide which can be tolerated and which should be addressed; for the latter, ethics sub-requirements should be developed to make changes to the design of the system, and so ensure the development of a more ethically aligned technology. Deciding which changes to prioritise is the most difficult part of the process, because there is no formulaic approach to follow; rather, this process requires interdisciplinary and multi-stakeholder discussions, individual and collective judgment, and ethics deliberation, which should include consideration of opportunity costs. However, that extends beyond the scope of this paper. Below we only give examples of how the matrices could be used to develop sub-requirements to specify RQ1-4. In line with the core aim of this paper, we only give examples focusing on the swarm's emergent behaviour.
%Building on \cite{Porter2022}, and adapting it for our purposes, we developed ethics requirements for swarm robots around the ethical principles of beneficence, non-maleficence, respect for autonomy, and justice. To limit the scope of this paper, we decided to focus on the impact of our case study on members of the public, trained humans, manufacturers, and the company deploying the swarm, hence leaving aside questions about broader societal and environmental implications.
%
% The main ethics requirements, each corresponding to a principle, are shown in Table \ref{tab:ethics} below. 
% \begin{table}[!h]
%	 	\centering
%	 	\begin{tabular}{|p{10mm}|p{112mm}|}
	%		 		\hline
	%		 		& \textbf{ Ethics Requirements} \\
	%		 		\hline
	%		 		RQ1 & The swarm \emph{shall} provide benefits to stakeholders. \\ 
	%		 		\hline
	%		 		RQ2 & The swarm \emph{shall} avoid causing unjustified harm to stakeholders. \\ 
	%		 		\hline
	%		 		RQ3 & The swarm \emph{shall} avoid undue constraints on stakeholders' personal autonomy. \\
	%		 		\hline
	%		 		RQ4 & The swarm \emph{shall} allow fair treatment of stakeholders.  \\		[1ex] 		
	%		 		\hline
	%		 	\end{tabular}
%	 	\caption{\label{tab:ethics}Ethics requirements for the cloakroom.}
%	 \end{table}  
%
%These broad requirements should be further specified, and this is where developing ethics matrices is helpful. The matrices allow us to identify benefits, risks of harm, and threats to autonomy and fair treatment caused by the development and deployment of the system. The significance and likelihood of the identified benefits, harms, and threats should then be considered to decide which can be tolerated and which should be addressed; for the latter, ethics sub-requirements should be developed to make changes to the design of the system, and so ensure the development of a more ethically aligned technology. Deciding which changes to prioritise is the most difficult part of the process, because there is no formulaic approach to follow; rather, this process requires interdisciplinary and multi-stakeholder discussions, individual and collective judgment, and ethics deliberation, which should include consideration of opportunity costs. However, that extends beyond the scope of this paper. Below we only give examples of how the matrices could be used to develop sub-requirements to specify RQ1-4. In line with the core aim of this paper, we only give examples focusing on the swarm's emergent behaviour.

% BEGINING OF PETE'S ADDITION

%\subsubsection{Using the Structural, Organisational, Technological, Epistemic, and Cultural (SOTEC) framework for a regulatory requirements analysis of autonomous robotic swarms}\label{regulations-r}

%In addition to ethical requirements, we also introduce regulatory requirements into the consideration of specification produced in Stage 2.
%In particular we have observed the work of Macrae \cite{macrae2021learning}, who proposes: Structural, Organisational, Technological, Epistemic, and Cultural (SOTEC) framework of sociotechnical risk in Autonomous and Intelligent systems (AIS) and discusses its regulatory implications for autonomous swarms in a cloakroom scenario. The SOTEC framework proposes that risk and failure in AIS be understood in terms of five broad sociotechnical categories, each corresponding to different organisational, contextual, and human factors that might inform AIS safety requirements. The spotlight on sociotechnical sources of risk in AIS throughout section one provides a crucial launching point for the identification of regulatory requirements. Viewing regulatory requirement analysis from a sociotechnical perspective allows us to move away from a purely technical conception of requirements, and helps us design autonomous systems that better fit the organisation and operators’ work in which safety considerations are meaningful within the wider system and operational context. 
%%
%Based on these considerations, we illustrate the relevance of the SOTEC framework for crafting regulatory requirements for autonomous robotic swarms as a safety assurance mechanism. Following this, we then apply the framework by utilising insights from swarm engineers who are in the process of developing a case study of a swarm system for a cloakroom setting that is in early development and offering autonomous solutions to customers depositing their garments.
% END OF PETE'S ADDITIONS

\vspace{-2ex}
\section*{Acknowledgments}
\vspace{-2.5ex}
The work presented in this paper has been supported by the UK Engineering and Physical Sciences Research Council (EPSRC) under the grant [EP/V026518/1].

\vspace{-1ex}
%
% ---- Bibliography ----
%
% BibTeX users should specify bibliography style 'splncs04'.
% References will then be sorted and formatted in the correct style.
%
%\begin{spacing}{1}
\scriptsize
\bibliographystyle{splncs04}
\bibliography{AERoS-Bib}%{AssuranceFWK-Swarms-Bibliography}
%\end{spacing}
%
\end{document}